\documentclass[a4paper,twoside,addpoints]{exam}
\usepackage{fontspec}
\usepackage[french]{babel}

\usepackage{amsmath}
\usepackage{xfrac}
\usepackage{tabularx}
\usepackage{siunitx}
\usepackage{qrcode}
\usepackage{float}
\usepackage{tabulary}
\usepackage{xpatch}
\usepackage{xcolor}
\usepackage{tikz}
\usepackage{realboxes}
\usepackage{parskip}
\usepackage{inconsolata}
\usepackage[framemethod=tikz]{mdframed}
\usepackage[most,xparse,skins,breakable]{tcolorbox}
\usetikzlibrary{calc}
\usepackage{tabularx,ragged2e}
\usepackage{zref-savepos}
\usepackage{heiglogo}
\usepackage{bashful}
\usepackage{xstring}
\usepackage{booktabs}

\def\theyear{2026}
\def\thedate{2026-05-01}

\def\documentType{Travail Écrit}

\def\department{TIN}
\def\school{HEIG-VD}
\def\duration{90 minutes}
\def\orientations{Informatique}
\def\course{Programmation C}
\def\examtype{TE}
\def\commit{}
\def\commiturl{}

\title{\documentType{} \course{} \orientations}


\author{Prof. Yves Chevallier}

\vtword{Total}
\vqword{Problème}

\setlength{\gridsize}{4mm}
\setlength{\gridlinewidth}{0.15pt}

\definecolor{GithubGray}{rgb}{0.9,0.9,0.9}
\definecolor{FillWithDottedLinesColor}{gray}{0.6}

\colorfillwithdottedlines

\newtcolorbox{examrules}[1][]{%
  enhanced jigsaw,
  sharp corners,
  colback=white,
  borderline={1pt}{-2pt}{black},
  #1
}

\lstset{
    language=c,
    breaklines=true,
    keywordstyle=\bfseries\color{black},
    basicstyle=\ttfamily\color{black}\fontsize{9pt}{10pt}\selectfont,
    emphstyle={\em \color{gray}},
    keepspaces=true,
    showspaces=false,
    showtabs=false,
    tabsize=3,
    upquote=true,
    aboveskip=2pt,
    belowskip=2pt,
    framexleftmargin=0pt,
    extendedchars=true,
    literate=
    {à}{{\`a}}1
    {è}{{\`e}}1
    {é}{{\'e}}1
    {É}{{\'E}}1
    {ê}{{\^e}}1
    {î}{{\^i}}1
    {ö}{{\"o}}1
    {°}{\textdegree}1
    {ç}{{\c{c}}}1
}

\mdfsetup{
  roundcorner=4pt,
  innerleftmargin=4pt,
  innerrightmargin=4pt,
  innertopmargin=10pt,
  innerbottommargin=2pt,
  linecolor=black!60,
  linewidth=0.5pt,
  backgroundcolor=white!5,
  skipabove=1ex,
}

\newmdenv[innerleftmargin=4pt]{codeframe}


\newcommand{\CD}[1]{\Colorbox{GithubGray}{\lstinline{#1}}}

\makeatletter
\def\@maketitle{%
  \newpage
  \heiglogo[color=black]
  \null
  \vskip 10em%
  \begin{center}%
  \let \footnote \thanks
    {\LARGE \@title \par}%
    \ifprintanswers
    \large \textbf{Solution}
    \fi
    \vskip 1.0em%
    {\itshape\department\par}%
    \vskip .2em%
    {\itshape\school\par}%
    \vskip 3.0em%
    {
      \lineskip .2em%
      \begin{minipage}{0.8 \textwidth}%
      \begin{center}
      \begin{tabular}[t]{c}%
        \@author
      \end{tabular}\par
      \end{center}
    \end{minipage}
    }
    \vskip 1em%
    {\large \thedate}%
  \end{center}%
  \par
  \vskip 1.5em}

%% Headers and footers
\firstpageheader{\school~/~\ifprintanswers Solution \fi \documentType~\course}{}{\enspace\makebox[6cm]{}\bfseries Nom/Prénom :\enspace\makebox[6cm]{\hrulefill}}
\runningheader{\school~/~\ifprintanswers Solution \fi \documentType~\course}{}{\enspace\makebox[6cm]{}\bfseries Nom/Prénom :\enspace\makebox[6cm]{\hrulefill}}

\footer{\rightmark}
  {Page \thepage\ sur \pageref{LastPage}}
  {\course}%{\iflastpage{Fin du travail.}{Aller à la page suivante\ldots}}

\pagestyle{foot}

% Configure the question title
\qformat{%
  \xdef\rightmark{Problème \thenumquestions~-- \thequestiontitle}%
  \Large\textbf{Problème \thenumquestions~-- \thequestiontitle}%
  \quad ~ \large \emph{(\thepoints)} \hfill
  \vrule depth 1.5em width 0pt
  \thispagestyle{headandfoot}%
}

\newcommand*{\cleartoleftpage}{%
  \clearpage
    \if@twoside
    \ifodd\c@page
    \thispagestyle{foot}
      \hbox{}\newpage
      \if@twocolumn
      \thispagestyle{foot}
        \hbox{}\newpage
      \fi
    \fi
  \fi
}
\makeatother

\newcommand{\problemcover}[3]{
  \thispagestyle{foot}
  {\null
  \vskip 10em%
  \begin{center}%
    {\LARGE Problème #1 \par}%
    \vskip 1.0em%
    {\Large #2 \par}%
    \vskip 5em%
    \partialgradetable{#3}[v][questions]  \vspace*{10ex}
  \end{center}%
  \par
  \vskip 1.5em}
  \clearpage
}

\newcounter{NoTableEntry}
\renewcommand*{\theNoTableEntry}{NTE-\the\value{NoTableEntry}}

\newcommand*{\strike}[2]{%
  \multicolumn{1}{#1}{%
    \stepcounter{NoTableEntry}%
    \vadjust pre{\zsavepos{\theNoTableEntry t}}% top
    \vadjust{\zsavepos{\theNoTableEntry b}}% bottom
    \zsavepos{\theNoTableEntry l}% left
    \hspace{0pt plus 1filll}%
    #2% content
    \hspace{0pt plus 1filll}%
    \zsavepos{\theNoTableEntry r}% right
    \tikz[overlay]{%
      \draw
        let
          \n{llx}={\zposx{\theNoTableEntry l}sp-\zposx{\theNoTableEntry r}sp-\tabcolsep},
          \n{urx}={\tabcolsep},
          \n{lly}={\zposy{\theNoTableEntry b}sp-\zposy{\theNoTableEntry r}sp},
          \n{ury}={\zposy{\theNoTableEntry t}sp-\zposy{\theNoTableEntry r}sp}
        in
        (\n{llx}, \n{lly}) -- (\n{urx}, \n{ury})
      ;
      \draw
        let
          \n{llx}={\zposx{\theNoTableEntry l}sp-\zposx{\theNoTableEntry r}sp-\tabcolsep},
          \n{urx}={\tabcolsep},
          \n{lly}={\zposy{\theNoTableEntry b}sp-\zposy{\theNoTableEntry r}sp},
          \n{ury}={\zposy{\theNoTableEntry t}sp-\zposy{\theNoTableEntry r}sp}
        in
        (\n{llx}, \n{ury}) -- (\n{urx}, \n{lly})
      ;
    }%
  }%
}

\lstMakeShortInline|

\begin{document}
\unframedsolutions

\renewcommand{\solutiontitle}{}
\begin{coverpages}
  \maketitle
  \thispagestyle{empty}

  \begin{center}
    \noindent Durée : \duration~minutes
    \vfil
    %La note est calculée par la relation : $\text{note} = \sfrac{\text{total}}{10} + 1$.
  \end{center}

  \begin{center}
    \gradetable[v][questions]
  \end{center}
  \vfil
  \begin{center}
    \begin{examrules}
      \textbf{Consignes : }
      \vskip 1em%
      \begin{itemize}
        \item Écrire votre \textbf{nom} et votre \textbf{prénom} lisiblement sur chaque page.
        \item Écrire lisiblement, au stylo ou au crayon à papier gras.
        \item Répondre aux questions dans les zones appropriées.
        \item Relire toutes vos réponses avant de rendre votre travail.
        \item Vérifier que vous n'avez pas oublié de compléter une page de l'examen.
        \item Rendre toutes les feuilles, une feuille par problème.
        \item Rendre toutes les feuilles de brouillon ainsi que la page de couverture.
        \item Les réponses données sur les feuilles de brouillon ne sont ni acceptées ni corrigées.
        \item Aucun moyen de communication autorisé.
        \item Toutes les réponses concernent le langage C et son standard C17 (\textsf{ISO/IEC 9899\string:2018}).
      \end{itemize}
    \end{examrules}
  \end{center}
\end{coverpages}

\newpage
\thispagestyle{empty}
\null\vfill
\newpage
\setcounter{page}{1}

% Input all questions/problems
\begin{questions}
\titledquestion{Expressions}[10]

Considérez les déclarations suivantes :

\begin{codeframe}
\begin{lstlisting}
#include <math.h>
#include <limits.h>
#define M 64
double x, y, z;
int i, j, k; // 32 bits
unsigned char c;
\end{lstlisting}
\end{codeframe}

Supposez que toutes les variables ont été initialisées avec certaines valeurs valides puis pour chacun des énoncés suivants, construire une expression C valide adaptée. Aucun mot-clef de structure de contrôle n'est autorisé. \par

L'exemple ci-après donne une expression pour le discriminant d'une équation du second degré donnant 1 si le discriminant est positif ou nul, 0 sinon. 

\begin{codeframe}
\begin{lstlisting}
(y*y - 4*x*z) >= 0 && x != 0\end{lstlisting}
\end{codeframe}


\begin{parts}
\part L'expression retourne la moyenne géométrique de \CD{x}, \CD{y} et \CD{z} exprimée comme $\sqrt[3]{x \cdot y \cdot z}$. Notez que la racine cubique est équivalente à la puissance $\frac{1}{3}$.

\begin{solutionordottedlines}[1\dottedlinefillheight]
\begin{codeframe}
\begin{lstlisting}
pow(x * y * z, 1.0/3.0)
\end{lstlisting}
\end{codeframe}
\end{solutionordottedlines}

\part L'expression retourne la partie entière (arrondie à la valeur inférieure) du logarithme base 2 de la variable \CD{i} ; on suppose $i > 0$.
\begin{solutionordottedlines}[1\dottedlinefillheight]
\begin{codeframe}
\begin{lstlisting}
floor(log10(i)/log10(2))
\end{lstlisting}
\end{codeframe}
\end{solutionordottedlines}

\part L'expression est vraie \underline{si et seulement si} la somme de \CD{x}, \CD{y} et \CD{z} est comprise strictement entre $10$ et $20$.

\begin{solutionordottedlines}[1\dottedlinefillheight]
\begin{codeframe}
\begin{lstlisting}
(x + y + z) > 10 && (x + y + z) < 20\end{lstlisting}
\end{codeframe}
\end{solutionordottedlines}

\part L'expression retourne la taille, en \underline{bits}, du type \CD{int} tel que défini sur le système.

\begin{solutionordottedlines}[1\dottedlinefillheight]
\begin{codeframe}
\begin{lstlisting}
sizeof(int) * CHAR_BIT\end{lstlisting}
\end{codeframe}
\end{solutionordottedlines}

\part L'expression est vraie si \CD{i} et \CD{j} sont impaires et que \CD{k} est un multiple de \CD{j}.

\begin{solutionordottedlines}[1\dottedlinefillheight]
\begin{codeframe}
\begin{lstlisting}
i % 2 != 0 && j % 2 != 0 && k % j == 0\end{lstlisting}
\end{codeframe}
\end{solutionordottedlines}

\part L'expression retourne la somme de \CD{i} et \CD{j}, mais après son évaluation, \CD{i} et \CD{j} ont toutes les deux été décrémentées de $1$. N'utilisez que les opérateurs \CD{--} et \CD{+}.

\begin{solutionordottedlines}[1\dottedlinefillheight]
\begin{codeframe}
\begin{lstlisting}
i-- + j--\end{lstlisting}
\end{codeframe}
\end{solutionordottedlines}

\part L'expression calcule la partie fractionnaire de $\frac{x}{y}$ ; on suppose que $y \neq 0$.

\begin{solutionordottedlines}[1\dottedlinefillheight]
\begin{codeframe}
\begin{lstlisting}
x / y - (int)(x / y)\end{lstlisting}
\end{codeframe}
\end{solutionordottedlines}

\part L'expression est vraie \underline{si et seulement si} le code ASCII contenu dans \CD{c} correspond à un caractère dans les intervalles allant de \CD{'0'} à \CD{'9'} ou de \CD{'a'} à \CD{'f'}, toutes bornes incluses.

\begin{solutionordottedlines}[1\dottedlinefillheight]
\begin{codeframe}
\begin{lstlisting}
(c >= '0' && c <= '9') || (c >= 'a' && c <= 'f')\end{lstlisting}
\end{codeframe}
\end{solutionordottedlines}

\part L'expression retourne la valeur des 4 bits de poids fort de l'inverse (bit à bit) de la variable \CD{c} ; la valeur doit être ramenée à l'intervalle compris entre 0 et 15 inclus.

\begin{solutionordottedlines}[1\dottedlinefillheight]
\begin{codeframe}
\begin{lstlisting}
(~c >> 4) & 0xF\end{lstlisting}
\end{codeframe}
\end{solutionordottedlines}

\part L'expression est vraie si les variables \CD{x}, \CD{x} et \CD{y} sont strictement dans un ordre décroissant soit $x > y > z$.

\begin{solutionordottedlines}[1\dottedlinefillheight]
\begin{codeframe}
\begin{lstlisting}
x > y && y > z\end{lstlisting}
\end{codeframe}
\end{solutionordottedlines}

\end{parts}

\end{questions}

% End of cover pages
\clearpage
\thispagestyle{empty}
\null
\label{LastPage}
\clearpage
\thispagestyle{empty}
\null
\vfill

\begin{center}
  Fin de l'examen.
\end{center}
\vskip 2em
\raggedleft\commit\quad\qrcode{\commiturl}

\end{document}

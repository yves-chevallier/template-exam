\titledquestion{Expressions}[10]

Considérez les déclarations suivantes :

\begin{codeframe}
\begin{lstlisting}
#include <math.h>
#include <limits.h>
#define M 64
double x, y, z;
int i, j, k; // 32 bits
unsigned char c;
\end{lstlisting}
\end{codeframe}

Supposez que toutes les variables ont été initialisées avec certaines valeurs valides puis pour chacun des énoncés suivants, construire une expression C valide adaptée. Aucun mot-clef de structure de contrôle n'est autorisé. \par

L'exemple ci-après donne une expression pour le discriminant d'une équation du second degré donnant 1 si le discriminant est positif ou nul, 0 sinon. 

\begin{codeframe}
\begin{lstlisting}
(y*y - 4*x*z) >= 0 && x != 0\end{lstlisting}
\end{codeframe}


\begin{parts}
\part L'expression retourne la moyenne géométrique de \CD{x}, \CD{y} et \CD{z} exprimée comme $\sqrt[3]{x \cdot y \cdot z}$. Notez que la racine cubique est équivalente à la puissance $\frac{1}{3}$.

\begin{solutionordottedlines}[1\dottedlinefillheight]
\begin{codeframe}
\begin{lstlisting}
pow(x * y * z, 1.0/3.0)
\end{lstlisting}
\end{codeframe}
\end{solutionordottedlines}

\part L'expression retourne la partie entière (arrondie à la valeur inférieure) du logarithme base 2 de la variable \CD{i} ; on suppose $i > 0$.
\begin{solutionordottedlines}[1\dottedlinefillheight]
\begin{codeframe}
\begin{lstlisting}
floor(log10(i)/log10(2))
\end{lstlisting}
\end{codeframe}
\end{solutionordottedlines}

\part L'expression est vraie \underline{si et seulement si} la somme de \CD{x}, \CD{y} et \CD{z} est comprise strictement entre $10$ et $20$.

\begin{solutionordottedlines}[1\dottedlinefillheight]
\begin{codeframe}
\begin{lstlisting}
(x + y + z) > 10 && (x + y + z) < 20\end{lstlisting}
\end{codeframe}
\end{solutionordottedlines}

\part L'expression retourne la taille, en \underline{bits}, du type \CD{int} tel que défini sur le système.

\begin{solutionordottedlines}[1\dottedlinefillheight]
\begin{codeframe}
\begin{lstlisting}
sizeof(int) * CHAR_BIT\end{lstlisting}
\end{codeframe}
\end{solutionordottedlines}

\part L'expression est vraie si \CD{i} et \CD{j} sont impaires et que \CD{k} est un multiple de \CD{j}.

\begin{solutionordottedlines}[1\dottedlinefillheight]
\begin{codeframe}
\begin{lstlisting}
i % 2 != 0 && j % 2 != 0 && k % j == 0\end{lstlisting}
\end{codeframe}
\end{solutionordottedlines}

\part L'expression retourne la somme de \CD{i} et \CD{j}, mais après son évaluation, \CD{i} et \CD{j} ont toutes les deux été décrémentées de $1$. N'utilisez que les opérateurs \CD{--} et \CD{+}.

\begin{solutionordottedlines}[1\dottedlinefillheight]
\begin{codeframe}
\begin{lstlisting}
i-- + j--\end{lstlisting}
\end{codeframe}
\end{solutionordottedlines}

\part L'expression calcule la partie fractionnaire de $\frac{x}{y}$ ; on suppose que $y \neq 0$.

\begin{solutionordottedlines}[1\dottedlinefillheight]
\begin{codeframe}
\begin{lstlisting}
x / y - (int)(x / y)\end{lstlisting}
\end{codeframe}
\end{solutionordottedlines}

\part L'expression est vraie \underline{si et seulement si} le code ASCII contenu dans \CD{c} correspond à un caractère dans les intervalles allant de \CD{'0'} à \CD{'9'} ou de \CD{'a'} à \CD{'f'}, toutes bornes incluses.

\begin{solutionordottedlines}[1\dottedlinefillheight]
\begin{codeframe}
\begin{lstlisting}
(c >= '0' && c <= '9') || (c >= 'a' && c <= 'f')\end{lstlisting}
\end{codeframe}
\end{solutionordottedlines}

\part L'expression retourne la valeur des 4 bits de poids fort de l'inverse (bit à bit) de la variable \CD{c} ; la valeur doit être ramenée à l'intervalle compris entre 0 et 15 inclus.

\begin{solutionordottedlines}[1\dottedlinefillheight]
\begin{codeframe}
\begin{lstlisting}
(~c >> 4) & 0xF\end{lstlisting}
\end{codeframe}
\end{solutionordottedlines}

\part L'expression est vraie si les variables \CD{x}, \CD{x} et \CD{y} sont strictement dans un ordre décroissant soit $x > y > z$.

\begin{solutionordottedlines}[1\dottedlinefillheight]
\begin{codeframe}
\begin{lstlisting}
x > y && y > z\end{lstlisting}
\end{codeframe}
\end{solutionordottedlines}

\end{parts}

\begingradingrange{Programme complet}
\titledquestion{Programme complet}[20]

Le programme \CD{spirale} génère une matrice carrée de taille \CD{WIDTH} et la remplit en spirale en partant du centre. La valeur de départ vaut \CD{1} et s'incrémente au fur et à mesure du parcours dans la spirale. La matrice est ensuite affichée sur la sortie standard. Voici un exemple d'utilisation pour \CD{#define WIDTH 5} ; observez bien la logique de placement des nombres dans la matrice : \par

\begin{codeframe}
\begin{lstlisting}
$ ./spirale
  17   16   15   14   13 
  18    5    4    3   12 
  19    6    1    2   11 
  20    7    8    9   10 
  21   22   23   24   25 
\end{lstlisting}
\end{codeframe}

Le programme se compose de la fonction \CD{main} qui doit appeler les deux fonctions suivantes :

\begin{enumerate}
\item \CD{void fill(int m[WIDTH][WIDTH])} qui remplit la matrice \CD{m} en spirale en partant du centre,
\item \CD{void display(int m[WIDTH][WIDTH])} qui affiche la matrice \CD{m} sur la sortie standard.
\end{enumerate}

Vous pouvez éventuellement vous aider du tableau de direction fourni ci-dessous. Il contient les déplacements relatifs (ligne, colonne) pour se déplacer d'une cellule à l'autre en fonction de la direction voulue :

\begin{codeframe}
\begin{lstlisting}
int directions[][2] = {
    {0, 1},  // EST   (droite)
    {-1, 0}, // NORD  (haut)
    {0, -1}, // OUEST (gauche)
    {1, 0},  // SUD   (bas)
};
\end{lstlisting}
\end{codeframe}

\textbf{Cahier des charges :}\\

\begin{enumerate}
\item le programme \textbf{doit} être complet (include, main, fonctions, etc.),
\item \CD{WIDTH} est une macro sans paramètre définie dans le programme,
\item on suppose que \CD{WIDTH} est impaire,
\item la fonction d'affichage se comporte comme l'exemple (alignement des valeurs),
\item la valeur de \CD{WIDTH} maximale est 99, soit un affichage sur 4 chiffres maximum.
\end{enumerate}

\newpage
\begin{solutionordottedlines}[\stretch{1}]
\begin{codeframe}
\lstinputlisting{assets/spirale.c}
\end{codeframe}

\vfill
\begin{center}
\begin{tabular}{ccl} \toprule
\multicolumn{3}{c}{Critères d'évaluation} \\ \midrule
\textbf{Points} & \textbf{Contexte} & \textbf{Critère} \\ \midrule
2 & \CD{fill} & Utilisation du modulo \\
2 & \CD{fill} & Utilisation de la variable \CD{WIDTH} \\
3 & \CD{fill} & Logique du changement de direction \\
1 & \CD{fill} & Condition de sortie de la boucle \\
1 & \CD{fill} & Condition de départ \\
2 & \CD{display} & Format de l'affichage \\
2 & \CD{display} & Boucle imbriquée \\
1 & \CD{display} & Condition de sortie \\
3 & \CD{main} & Appel des deux fonctions et prototype correct \\
3 & \CD{main} & Constantes et includes \\
\end{tabular}
\end{center}

\end{solutionordottedlines}

\ifprintanswers
\else
\newpage
\fillwithdottedlines{\stretch{1}}
\newpage
\fillwithdottedlines{\stretch{1}}
\fi
\endgradingrange{Programme complet}

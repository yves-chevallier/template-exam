\titledquestion{Fonctions}[15]

Écrire des fonctions C complètes correspondantes aux descriptions. Faire l'hypothèse que toutes les bibliothèques standards ont été importées en amont. L'exemple est donné pour la fonction \CD{hello} qui affiche \CD{Hello World!} sur la sortie standard.

\begin{codeframe}
\begin{lstlisting}
void hello() { printf("Hello World!\n"); }
\end{lstlisting}
\end{codeframe}

\begin{parts}

\part La fonction \CD{manhattan\_distance} s'applique à des \underline{entiers}. Elle calcule et retourne la distance de Manhattan entre deux points 2D.
Les coordonnées de chaque point $(x_1, y_1)$ et $(x_2, y_2)$ sont reçues en paramètres. La distance de Manhattan vaut $\lvert x2 - x1\rvert + \lvert y2 - y1\rvert$.
\begin{solutionordottedlines}[8\dottedlinefillheight]
\begin{codeframe}
\begin{lstlisting}
int manhattan_distance(int x1, int y1, int x2, int y2) {
    return abs(x2 - x1) + abs(y2 - y1);
}\end{lstlisting}
\end{codeframe}
\end{solutionordottedlines}

\part La fonction \CD{div_mod} reçoit deux entiers \CD{a} et \CD{b} passés par adresse. Si la valeur pointée par \CD{b} est nulle, la fonction retourne \CD{true}. Sinon elle remplace respectivement les valeurs pointées par \CD{a} et \CD{b} par le quotient et le reste de la division de 
\CD{a} par \CD{b}. La fonction retourne alors \CD{false}.


\begin{solutionordottedlines}[10\dottedlinefillheight]
\begin{codeframe}
\begin{lstlisting}
bool div_mod(int *a, int *b) {
    if (*b == 0) return true;
    int tmp = *a;
    *a = *a / *b;
    *b = tmp % *b;
    return false;
}\end{lstlisting}
\end{codeframe}
\end{solutionordottedlines}

\newpage
\part La fonction \CD{bit_set} retourne le nombre de bits à 1 dans la représentation binaire d'un entier non signé \CD{n} strictement codé sur 32 bits reçu en paramètre (type \CD{uint32_t}).
\begin{solutionordottedlines}[8\dottedlinefillheight]
\begin{codeframe}
\begin{lstlisting}
int bit_set(uint32_t n) {
    int count = 0;
    const length = sizeof(n) * CHAR_BIT;
    for (int i = 0; i < length; i++)
        count += n & (1 << i);
    return count;
}\end{lstlisting}
\end{codeframe}
\end{solutionordottedlines}

\part La fonction \CD{max_diff} retourne la différence entre le plus grand et le plus petit élément d'un tableau d'entiers \CD{arr} reçu en paramètre. Le nombre d'éléments \CD{n} du tableau est reçu comme deuxième paramètre ; il est supposé supérieure à 0.

\begin{solutionordottedlines}[10\dottedlinefillheight]
\begin{codeframe}
\begin{lstlisting}
int max_diff(int *arr, int n) {
    int min = arr[0], max = arr[0];
    for (int i = 1; i < n; i++) {
        if (arr[i] < min) min = arr[i];
        if (arr[i] > max) max = arr[i];
    }
    return max - min;
}\end{lstlisting}
\end{codeframe}
\end{solutionordottedlines}

\part Une fonction \CD{check} reçoit une chaîne de caractères \CD{str} de la forme \CD{"12 -23 -42"} où les \underline{trois} nombres \underline{entiers} peuvent être positifs ou négatifs. La fonction retourne \CD{true} si la chaîne est valide et que les trois nombres sont différents les uns des autres. Sinon, elle retourne \CD{false}. Les éventuels caractères situés après le troisième nombre ne sont pas pris en compte.

\begin{solutionordottedlines}[12\dottedlinefillheight]
\begin{codeframe}
\begin{lstlisting}
bool check(const char *str) {
    int a, b, c;
    if (sscanf(str, "%d %d %d", &a, &b, &c) != 3) return false;
    return a != b && a != c && b != c;
}\end{lstlisting}
\end{codeframe}
\end{solutionordottedlines}

\end{parts}

\newcommand{\crosstab}{\ifprintanswers \strike{c|}{\hspace{2em}} \fi}

\newdimen\colwidth
\setlength\colwidth{25pt}

\titledquestion{Compréhension}[5]

Pour chaque portion d'un programme ci-dessous, \textbf{indiquez} ce qui s'affiche sur la sortie standard. \textbf{Marquez la fin} de l'affichage avec une \underline{croix} comme précisé dans l'exemple. S'il n'y a pas d'affichage ou une boucle infinie, veuillez en \textbf{indiquer les raisons}.\par

% Correspondances ASCII utiles :

% \begin{tabularx}{\linewidth}{*{8}{>{\Centering}X}}
%     \toprule
%     \texttt{'a'} & \texttt{'z'} & \texttt{'A'} & \texttt{'Z'} & \texttt{'0'} & \texttt{'9'} & \texttt{EOF} & \texttt{'\textbackslash{}n'} \\
%     \midrule
%     97 & 122 & 65 & 90 & 48 & 57 & 255 & 10 \\
%     0x61 & 0x7A & 0x41 & 0x5A & 0x30 & 0x39 & 0xFF & 0x0A \\
%     \bottomrule
% \end{tabularx}

Exemple :

\begin{codeframe}
\begin{lstlisting}
int val = 'a';
printf("Val=%d", val);\end{lstlisting}
\end{codeframe}
\hfill {\tabcolsep=0pt\def\arraystretch{1.3}
\begin{tabularx}{9\colwidth}{|*9{>{\Centering}X|}}\hline
    \texttt{V} &
    \texttt{a} &
    \texttt{l} &
    \texttt{=} &
    \texttt{9} &
    \texttt{7} &
    \strike{c|}{\hspace{2em}} &  &
    \tabularnewline
    \hline
\end{tabularx}
\vskip 1em}

\begin{parts}

\part
\begin{codeframe}
\begin{lstlisting}
float x = 2.18281828459045;
char a = 'a';
printf("%1.0f%hhd%%", x, a);
\end{lstlisting}
\end{codeframe}
\hfill \partlabel {\tabcolsep=0pt\def\arraystretch{1.3}
\begin{tabularx}{8\colwidth}{|*8{>{\Centering}X|}}\hline
    \fillin[\texttt{2}][0in] &
    \fillin[\texttt{9}][0in] &
    \fillin[\texttt{7}][0in] &
    \fillin[\texttt{\%}][0in] &
    \crosstab &  &  &
    \tabularnewline
    \hline
\end{tabularx}\vskip 1em}

\part
\begin{codeframe}
\begin{lstlisting}
char *str = "14:23:58";
int u, v, w;
printf("%s", sscanf(str, "%d:%d:%d", &u, &v, &w) == 3 ? "OK" : "KO");
\end{lstlisting}
\end{codeframe}
\hfill \partlabel {\tabcolsep=0pt\def\arraystretch{1.3}
\begin{tabularx}{8\colwidth}{|*8{>{\Centering}X|}}\hline
    \fillin[\texttt{O}][0in] &
    \fillin[\texttt{K}][0in] &
    \crosstab & & & & &
    \tabularnewline
    \hline
\end{tabularx}\vskip 1em}

\part
\begin{codeframe}
\begin{lstlisting}
unsigned int a[] = {4, 8, 42, 23};
const int s = sizeof(a) / sizeof(a[0]);
for (int i = 0; i < s - 1; ++i) {
    if (a[i] <= a[i + 1]) continue;
    int t = a[i];
    a[i] = a[i + 1];
    a[i + 1] = t;
}
for (int i = 0; i < s; i++) printf("%u ", a[i]);\end{lstlisting}
\end{codeframe}
\hfill \partlabel {\tabcolsep=0pt\def\arraystretch{1.3}
\begin{tabularx}{14\colwidth}{|*{14}{>{\Centering}X|}}\hline
    \fillin[\texttt{4}][0in] &
    \fillin[\texttt{ }][0in] &
    \fillin[\texttt{8}][0in] &
    \fillin[\texttt{ }][0in] &
    \fillin[\texttt{2}][0in] &
    \fillin[\texttt{3}][0in] &
    \fillin[\texttt{ }][0in] &
    \fillin[\texttt{4}][0in] &
    \fillin[\texttt{2}][0in] &
    \fillin[\texttt{ }][0in] &
    \crosstab & & &
    \tabularnewline
    \hline
\end{tabularx}\vskip 1em}

\part
\begin{codeframe}
\begin{lstlisting}
int u = 15421 ^ 15421;
switch (u) {
    case 0:
    case 1:
        printf("1");
    case 2:
        printf("2");
        break;
    case 3:
        printf("3");
    default:
        printf("42");
}
\end{lstlisting}

    \end{codeframe}
    \hfill \partlabel {\tabcolsep=0pt\def\arraystretch{1.3}
    \begin{tabularx}{8\colwidth}{|*8{>{\Centering}X|}}\hline
        \fillin[\texttt{1}][0in] &
        \fillin[\texttt{2}][0in] &
        \crosstab & & & & &
        \tabularnewline
        \hline
    \end{tabularx}\vskip 1em}

\part
\begin{codeframe}
\begin{lstlisting}
float f = 3.1415926535;
while(f > 3.0f) {
    printf("%05.2f", f);
    f -= 1.0f;
}
\end{lstlisting}

    \end{codeframe}
    \hfill \partlabel {\tabcolsep=0pt\def\arraystretch{1.3}
    \begin{tabularx}{8\colwidth}{|*8{>{\Centering}X|}}\hline
        \fillin[\texttt{0}][0in] &
        \fillin[\texttt{3}][0in] &
        \fillin[\texttt{.}][0in] &
        \fillin[\texttt{1}][0in] &
        \fillin[\texttt{4}][0in] &
        \crosstab & &
        \tabularnewline
        \hline
    \end{tabularx}\vskip 1em}

\end{parts}
